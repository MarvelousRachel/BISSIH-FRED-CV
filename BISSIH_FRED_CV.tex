
\documentclass[11pt,a4paper]{article}
\usepackage[utf8]{inputenc}
\usepackage[margin=1in]{geometry}
\usepackage{enumitem}
\usepackage{titlesec}
\usepackage{xcolor}
\usepackage{hyperref}
\usepackage{fontawesome}

% Define colors
\definecolor{darkblue}{RGB}{0,51,102}
\definecolor{lightgray}{RGB}{128,128,128}

% Configure hyperlinks
\hypersetup{
    colorlinks=true,
    linkcolor=darkblue,
    urlcolor=darkblue,
    emailcolor=darkblue
}

% Configure section formatting
\titleformat{\section}{\Large\bfseries\color{darkblue}}{}{0em}{}[\titlerule]
\titleformat{\subsection}{\large\bfseries}{}{0em}{}

% Remove page numbers
\pagestyle{empty}

\begin{document}

% Header with name and contact info
\begin{center}
{\Huge\bfseries BISSIH FRED}\\[0.3cm]
{\color{lightgray}\faMapMarker} China, Guangdong Province, Zhanjiang, Mazhang District, 海大路1号 邮政编码: 524091\\
{\color{lightgray}\faEnvelope} \href{mailto:fredbissih@gmail.com}{fredbissih@gmail.com}, \href{mailto:1252201246@stu.gdou.edu.cn}{1252201246@stu.gdou.edu.cn}\\
{\color{lightgray}\faPhone} +86 132 4651 6503
\end{center}

\vspace{0.5cm}

\section{CAREER OBJECTIVE}
Innovative and results-driven aquaculture nutritionist and feed technologist with strong academic and industrial experience in fish feed formulation, nutritional requirement studies, and aquaculture systems. Proven expertise in developing cost-effective feeds using alternative protein sources and conducting large-scale feeding trials. Seeking to leverage advanced knowledge in feed technology to improve fish growth, sustainability, and profitability in Kenya's aquafeed industry.

\section{EDUCATION}

\subsection{PhD in Aquaculture Science (Final Year, 4th Year)}
\textbf{Fisheries College, Guangdong Ocean University – Zhanjiang, China}
\begin{itemize}[leftmargin=*,noitemsep]
    \item Supervisor: Prof. Shuyan Chi
    \item Research Focus: Nutritional requirements of marine fish (Golden pompano) with emphasis on high lipid and amino acid nutrition
    \item Expected Completion: 2025
\end{itemize}

\subsection{MSc in Aquaculture Science}
\textbf{Chinese Academy of Fishery Sciences / Huzhou University School of Life Science – Zhejiang, China}
\begin{itemize}[leftmargin=*,noitemsep]
    \item Supervisor: Prof. Chenlong Wu
    \item Thesis: Successful replacement of fishmeal with low gossypol cottonseed meal in the diet of Largemouth Bass (\textit{Micropterus salmoides})
    \item Graduated with Distinction
\end{itemize}

\subsection{BSc in Natural Resources Management (Fisheries \& Aquaculture option)}
\textbf{University of Energy and Natural Resources – Sunyani, Ghana}
\begin{itemize}[leftmargin=*,noitemsep]
    \item Thesis: Successful replacement of fishmeal with Black Soldier Fly meal in the diet of Nile Tilapia (\textit{Oreochromis niloticus})
    \item Graduated with First Class Honors
\end{itemize}

\section{PROFESSIONAL \& RESEARCH EXPERIENCE}

\subsection{PhD Researcher – Aquatic Animal Nutrition \& Feed Laboratory}
\textbf{Guangdong Ocean University, China} \hfill (2021–Present)
\begin{itemize}[leftmargin=*,noitemsep]
    \item Designed and executed 7 comprehensive feeding experiments on marine fish nutritional requirements
    \item Conducted two-year sea cage feeding trials on Golden pompano, optimizing growth performance by 18\%
    \item Completed industrial internship at Guangdong Yuehai Feed Company, formulating novel commercial feeds for Golden pompano that reduced production costs by 12\%
    \item Developed innovative high-lipid feed formulations that improved feed conversion ratio by 0.3 points
    \item Published research findings in peer-reviewed journals on amino acid and lipid requirements
\end{itemize}

\subsection{MSc Researcher – Key Laboratory of Aquatic Animal Genetic Breeding and Nutrition}
\textbf{Chinese Academy of Fishery Sciences / Huzhou University, China} \hfill (2018–2020)
\begin{itemize}[leftmargin=*,noitemsep]
    \item Participated in advanced research on freshwater fish feed formulations and nutritional biochemistry
    \item Developed cost-effective protein replacement strategies for largemouth bass, achieving 80\% fishmeal replacement
    \item Optimized feed pellet extrusion parameters for improved water stability and palatability
    \item Collaborated with industry partners to translate research findings into commercial applications
\end{itemize}

\subsection{Teaching Assistant \& National Service Personnel}
\textbf{University of Energy and Natural Resources, Ghana} \hfill (2016–2017)
\begin{itemize}[leftmargin=*,noitemsep]
    \item Assisted in teaching aquaculture and fisheries management courses to 120+ undergraduate students
    \item Provided hands-on training in pond culture techniques and fish nutrition to 45 students
    \item Supervised practical sessions on water quality management and fish feed formulation
    \item Coordinated field trips to commercial aquaculture facilities for experiential learning
\end{itemize}

\subsection{Industrial Attachment – Aquatic Feed \& Nutrition Department}
\textbf{Aquaculture Research and Development Center (ARDEC), Akosombo, Ghana} \hfill (2015)
\begin{itemize}[leftmargin=*,noitemsep]
    \item Received professional training in feed formulation and production for tilapia and catfish under Dr. K. Anani
    \item Gained practical experience with the renowned ARDEC fish feed production process
    \item Assisted in quality control of raw materials and finished feed products
    \item Participated in feeding trials evaluating growth performance of different fish strains
\end{itemize}

\section{TECHNICAL SKILLS}

\subsection{Feed Formulation \& Production}
\begin{itemize}[leftmargin=*,noitemsep]
    \item Expert in formulation of freshwater and marine fish feeds using conventional and alternative protein sources
    \item Proficient with least-cost feed formulation software and techniques
    \item Experienced in extrusion technology and pellet production
    \item Skilled in feed quality assessment and stability testing
\end{itemize}

\subsection{Fish Nutrition \& Physiology}
\begin{itemize}[leftmargin=*,noitemsep]
    \item Strong expertise in amino acid and lipid requirements of various aquaculture species
    \item Advanced knowledge of digestive physiology and nutrient utilization in fish
    \item Experienced in nutritional biochemistry and metabolic pathway analysis
    \item Proficient in nutrient requirement studies and deficiency assessments
\end{itemize}

\subsection{Aquaculture Systems}
\begin{itemize}[leftmargin=*,noitemsep]
    \item Practical experience with pond culture, freshwater recirculating systems, and marine sea cage farming
    \item Knowledge of water quality management and environmental monitoring
    \item Skilled in fish health assessment and disease prevention through nutrition
\end{itemize}

\subsection{Research \& Data Analysis}
\begin{itemize}[leftmargin=*,noitemsep]
    \item Proficiency in experimental design for nutritional trials
    \item Expert in growth performance analysis and feed utilization metrics
    \item Skilled in biochemical assays and proximate analysis techniques
    \item Advanced statistical data interpretation using R and SPSS
\end{itemize}

\section{ADDITIONAL SKILLS}
\begin{itemize}[leftmargin=*,noitemsep]
    \item \textbf{Teamwork \& Leadership:} Proven ability to work effectively in multicultural research environments
    \item \textbf{Scientific Communication:} Strong scientific writing skills with experience in research dissemination
    \item \textbf{Teaching \& Mentoring:} Capacity to train junior researchers and convey complex concepts
    \item \textbf{International Collaboration:} Extensive experience working across cultural boundaries
    \item \textbf{Languages:} Fluent in English, intermediate Chinese, basic French
    \item \textbf{Personal Interests:} Professional swimming coach, footballer, pianist, guitarist (demonstrating discipline \& teamwork)
\end{itemize}

\section{REFEREES}
Available upon request

\end{document}
